\input{preambuloSimple.tex}
\usepackage{url}

\title{	
	\normalfont \normalsize
	\begin{figure}[htb]
		\centering
		\includegraphics[width=0.3\textwidth]{./imagenes/1}
	\end{figure}
	\textsc{\textbf{Estructura de Computadores} \\ Grado en Ingeniería Informática \\ 
	Curso 2018-2019} \\ [25pt] % Your university, school and/or department name(s)
	\begin{figure}[htb]
		\centering
		\includegraphics[width=0.15\textwidth]{./imagenes/2}
	\end{figure}
	\horrule{0.5pt} \\[0.4cm] % Thin top horizontal rule
	\huge Memoria Práctica 2. \\
	\huge Programaci\'on ensamblador x86-64 Linux.
	\\ % The assignment title
	\horrule{2pt} \\[0.5cm] % Thick bottom horizontal rule
}
\author{Félix Ramírez García  \\
\href{mailto:felixramirezgarcia@correo.ugr.es}{felixramirezgarcia@correo.ugr.es}} % Nombre y apellidos
\date{\normalsize\today} % Incluye la fecha actual

%----------------------------------------------------------------------------------------
% DOCUMENTO
%----------------------------------------------------------------------------------------

\begin{document}
	
	\maketitle % Muestra el Título
	
	\newpage %inserta un salto de página
	
	\tableofcontents % para generar el índice de contenidos
	
	\listoffigures % para generar índice de imágenes.
	
	\listoftables % para generar índice de tablas.
	
	\newpage
	
	%-----------------------------------------------------------------------
	%							Diario de trabajo
	%----------------------------------------------------------------------	
	\section[Diario de trabajo]{Diario de trabajo.}
		
	\begin{table}[htbp]
		\begin{center}
			\begin{tabular}{|l|l|}
				\hline
				Fecha & Tarea desarrollada \\
				\hline \hline
				03-10-2018 & Repaso de la parte A del guion de la práctica \\ \hline
				06-10-2018 & Terminado el ejercicio 5.1 y empezado el 5.2 \\ \hline
				09-10-2018 & Terminados los ejercicios 5.2 , 5.3 y 5.4 \\ \hline
				12-10-2018 & Pasar la batería de test a los ejercicios \\ \hline
				14-10-2018 & Redacción de este documento en látex y últimos retoques a los ejercicios \\ \hline
			\end{tabular}
			\caption{Diario de trabajo de la práctica.}
			\label{tabla:sencilla}
		\end{center}
	\end{table}
	
	%-----------------------------------------------------------------------
	%							codigo de los programas
	%-----------------------------------------------------------------------
	\section[Código de los ejercicios]{Código de los ejercicios}
	
	%-----------------------------------------------------------------------	%							5.1
	%-----------------------------------------------------------------------
	
	\subsection{5.1 Sumar N enteros \underline{sin}  signo de 32 bits sobre registros de 32 bits usando uno de ellos como acumulador de acarreos. (N=16)}
	
	\lstset{language=[x64]Assembler}
	\begin{lstlisting}[frame=single]  
.section .data
	.macro linea
		#.int 1,1,1,1
		#.int 0x0fffffff,0x0fffffff,0x0fffffff,0x0fffffff
		.int 0x10000000,0x10000000,0x10000000,0x10000000
	.endm
lista: .irpc i,1234
	linea
.endr
longlista:      .int   (.-lista)/4
resultado:      .quad 0x0123456701234567
formato:        .asciz  "suma = %lu = 0x%lx hex\n"

.section .text
main: .global  main
	mov     $lista, %rbx
	mov  longlista, %ecx
	call suma               # == suma(&lista, longlista);
	mov  %eax, resultado
	mov  %edx, resultado+4  # 4 bytes mas significativos
	
	mov   $formato, %rdi
	mov   resultado,%rsi
	mov   resultado,%rdx
	mov          $0,%eax    # varargin sin xmm
	call  printf            # == printf(formato, res, res);
	
	mov  resultado, %edi
	call _exit              # ==  exit(resultado)
	ret
suma:
	push     %rsi
	mov  $0, %eax
	mov  $0, %edx
	mov  $0, %rsi
bucle:
	add (%rbx,%rsi,4), %eax # realiza la adicion de los registros
	jc  moreindex           # salto condicional si hay acarreo
otrobucle:
	inc  %rsi               # incremento el indice
	cmp  %rsi,%rcx          # comparo con rcx
	jne  bucle              # salto a bucle si no son iguales
	pop  %rsi
	ret
moreindex:
	inc  %edx               # incremento el registro
	cmp  %edx,%edx          # comparo para forzar el salto a otrobucle
	je   otrobucle          # salto a otrobucle
	ret
	\end{lstlisting}
	
	%-----------------------------------------------------------------------	%							5.2
	%-----------------------------------------------------------------------
		
	\subsection{5.2 Sumar N enteros \underline{sin} signo de 32 bits sobre registros de 32 bits meidante extensión con ceros. (N=16)}
	
	\lstset{language=[x64]Assembler}
	\begin{lstlisting}[frame=single]  
.section .data
	.macro linea
		#.int 1,1,1,1
		#.int 0x0fffffff,0x0fffffff,0x0fffffff,0x0fffffff
		#.int 0x10000000,0x10000000,0x10000000,0x10000000
		#.int 0xffffffff,0xffffffff,0xffffffff,0xffffffff
		#.int -1,-1,-1,-1
		#.int 200000000,200000000,200000000,200000000
		#.int 300000000,300000000,300000000,300000000
		.int 5000000000,5000000000,5000000000,5000000000
	.endm
lista: .irpc i,1234
linea
.endr
longlista:      .int   (.-lista)/4
resultado:      .quad 0x0123456701234567
formato:        .asciz  "suma = %lu = 0x%lx hex\n"

.section .text
main: .global  main
	mov     $lista, %rbx
	mov  longlista, %ecx
	call suma               # == suma(&lista, longlista);
	mov  %eax, resultado
	mov  %edx, resultado+4  # 4 bytes mas significativos
	
	mov   $formato, %rdi
	mov   resultado,%rsi
	mov   resultado,%rdx
	mov          $0,%eax    # varargin sin xmm
	call  printf            # == printf(formato, res, res);
	
	mov  resultado, %edi
	call _exit              # ==  exit(resultado)
	ret
suma:
	push     %rsi           # indice
	mov  $0, %eax
	mov  $0, %edx
	mov  $0, %rsi
bucle:
	add (%rbx,%rsi,4), %eax # realiza la adicion de los registros
	adc  $0, %edx           # incrementa si bit CF de acarreo activo
	inc  %rsi               # incrementa el indice
	cmp  %rsi,%rcx          # realiza la comparativa
	jne  bucle              # salta a bucle si no es igual
	pop  %rsi
	ret
	\end{lstlisting}
	
	%-----------------------------------------------------------------------	%							5.3
	%-----------------------------------------------------------------------
	
	\subsection{5.3 Sumar N enteros \underline{con} signo de 32 bits sobre registros de 32 bits (mediante extension de signo). (N=16)}
	
	\lstset{language=[x64]Assembler}
	\begin{lstlisting}[frame=single]
.section .data
	.macro linea
		#.int -1,-1,-1,-1
		#.int 0x04000000,0x04000000,0x04000000,0x04000000
		#.int 0x08000000,0x08000000,0x08000000,0x08000000
		#.int 0x10000000,0x10000000,0x10000000,0x10000000
		#.int 0x7fffffff,0x7fffffff,0x7fffffff,0x7fffffff
		#.int 0x80000000,0x80000000,0x80000000,0x80000000
		#.int 0xf0000000,0xf0000000,0xf0000000,0xf0000000
		#.int 0xf8000000,0xf8000000,0xf8000000,0xf8000000
		#.int 0xf7ffffff,0xf7ffffff,0xf7ffffff,0xf7ffffff
		#.int 100000000,100000000,100000000,100000000
		#.int 200000000,200000000,200000000,200000000
		#.int 300000000,300000000,300000000,300000000
		#.int 2000000000,2000000000,2000000000,2000000000
		#.int 3000000000,3000000000,3000000000,3000000000
		#.int -100000000,-100000000,-100000000,-100000000
		.int -200000000,-200000000,-200000000,-200000000
		#.int -300000000,-300000000,-300000000,-300000000
		#.int -2000000000,-2000000000,-2000000000,-2000000000
		#.int -3000000000,-3000000000,-3000000000,-3000000000
	.endm
lista: .irpc i,1234
linea
.endr
longlista:      .int   (.-lista)/4
resultado:      .quad 0x0123456701234567
formato:        .asciz  "suma = %ld = 0x%lx hex\n"

.section .text
main: .global  main
	mov     $lista, %rbx
	mov  longlista, %ecx
	call suma               # == suma(&lista, longlista);
	mov  %eax, resultado
	mov  %edx, resultado+4  # 4 bytes mas significativos
	
	mov   $formato, %rdi
	mov   resultado,%rsi
	mov   resultado,%rdx
	mov          $0,%eax    # varargin sin xmm
	call  printf            # == printf(formato, res, res);
	
	mov  resultado, %edi
	call _exit              # ==  exit(resultado)
	ret
suma:
	mov  $0, %edi           # acumulador
	mov  $0, %ebp           # acumulador
	mov  $0, %esi           # indice
bucle:
	mov  (%ebx,%esi,4), %eax
	cltd                    # signed long to signed double long
	add  %eax,%edi          # realizar la adicion de los registros
	adc  %edx,%ebp          # realiza la adicion si CF acctivo
	inc  %esi               # incrementa el indice
	cmp  %esi,%ecx          # comparamos
	jne  bucle              # saltamos a bucle si no son iguales
	mov  %edi,%eax
	mov  %ebp,%edx
	ret
	\end{lstlisting}
		
	%-----------------------------------------------------------------------	%							5.4
	%-----------------------------------------------------------------------
	
	\subsection{5.4 Media y resto de N enteros \underline{con} signo de 32 bits calculada usando registros de 32 bits. (N=16)}
	
	\lstset{language=[x64]Assembler}
	\begin{lstlisting}[frame=single]  
.section .data
	.macro linea
		.int 1,2,1,2
		#.int -1,-2,-1,-2
		#.int 0x7fffffff,0x7fffffff,0x7fffffff,0x7fffffff
		#.int 0x80000000,0x80000000,0x80000000,0x80000000
		#.int 0xffffffff,0xffffffff,0xffffffff,0xffffffff
		#.int 2000000000,2000000000,2000000000,2000000000
		#.int 3000000000,3000000000,3000000000,3000000000
		#.int -2000000000,-2000000000,-2000000000,-2000000000
		#.int -3000000000,-3000000000,-3000000000,-3000000000
	.endm
lista: .irpc i,1234
	linea
.endr
longlista:      .int   (.-lista)/4
media:          .int 0x89ABCDEF
resto:          .int 0x01234567
formato:        .asciz  "media = %8d  resto = %8d \n"

.section .text
main: .global  main
	mov     $lista, %rbx
	mov  longlista, %ecx
	call suma               # == suma(&lista, longlista);
	mov  %eax, media
	mov  %edx, resto
	
	mov   $formato, %rdi
	mov   media,%rsi
	mov   resto,%rdx
	mov   $0,%eax           # varargin sin xmm
	call  printf            # == printf(formato, res, res);
	
	mov  media, %edi
	call _exit              # ==  exit(resultado)
	ret
suma:
	mov  $0, %edi           # acumulador
	mov  $0, %ebp           # acumulador
	mov  $0, %esi           # indice
bucle:
	mov  (%ebx,%esi,4), %eax
	cltd                    # signed long to signed double long
	add  %eax,%edi          # adicion de registros
	adc  %edx,%ebp          # adicion si bit CF activo
	inc  %esi               # incrementar indice
	cmp  %esi,%ecx          # comparar
	jne  bucle              # saltar a bucle si no son iguales
	mov  %edi,%eax
	mov  %ebp,%edx
	idiv %ecx               # realizar la division
	ret	
	\end{lstlisting}
	
	%-----------------------------------------------------------------------
	%							Pruebas de ejecución
	%-----------------------------------------------------------------------
	\section[Pruebas de ejecución]{Pruebas de ejecución}
	
	%-----------------------------------------------------------------------	%							5.1
	%-----------------------------------------------------------------------	

	\begin{table}[htbp]
		\begin{center}
			\begin{tabular}{|l|l|}
				\hline
				Input list (x16) & Output \\
				\hline \hline
				1 & suma = 16 = 0x10 hex \\ \hline
				0x0fffffff & suma = 4294967280 = 0xfffffff0 hex \\ \hline
				0x10000000 & suma = 4294967296 = 0x100000000 hex \\ \hline
			\end{tabular}
			\caption{Resultado bateria ejercicio 5.1}
			\label{tabla:sencilla}
		\end{center}
	\end{table}
	
	%-----------------------------------------------------------------------	%							5.2
	%-----------------------------------------------------------------------
	
	\begin{table}[htbp]
		\begin{center}
			\begin{tabular}{|l|l|}
				\hline
				Input list (x16) & Output \\
				\hline \hline
				1 & suma = 16 = 0x10 hex  \\ \hline
				0x0fffffff & suma = 4294967280 = 0xfffffff0 hex  \\ \hline
				0x10000000 & suma = 4294967296 = 0x100000000 hex \\ \hline
				0xffffffff & suma = 68719476720 = 0xffffffff0 hex \\ \hline
				-1 & suma = 68719476720 = 0xffffffff0 hex \\ \hline
				200000000 & suma = 3200000000 = 0xbebc2000 hex  \\ \hline
				300000000 & suma = 4800000000 = 0x11e1a3000 hex  \\ \hline
				5000000000 & suma = 11280523264 = 0x2a05f2000 hex \\ \hline
			\end{tabular}
			\caption{Resultado bateria ejercicio 5.2}
			\label{tabla:sencilla}
		\end{center}
	\end{table}
	
	%-----------------------------------------------------------------------	%							5.3
	%-----------------------------------------------------------------------
	
	\begin{table}[htbp]
		\begin{center}
			\begin{tabular}{|l|l|}
				\hline
				Input list (x16) & Output \\
				\hline \hline
				-1 & suma = -16 = 0xfffffffffffffff0 hex  \\ \hline
				0x04000000 & suma = 1073741824 = 0x40000000 hex  \\ \hline-
				0x08000000 & suma = 2147483648 = 0x80000000 hex  \\ \hline
				0x10000000 & suma = 4294967296 = 0x100000000 hex  \\ \hline
				0x7fffffff & suma = 34359738352 = 0x7fffffff0 hex  \\ \hline
				0x80000000 & suma = -34359738368 = 0xfffffff800000000 hex  \\ \hline
				0xf0000000 & suma = -4294967296 = 0xffffffff00000000 hex  \\ \hline
				0xf8000000 & suma = -2147483648 = 0xffffffff80000000 hex  \\ \hline
				0xf7000000 & suma = -2147483664 = 0xffffffff7ffffff0 hex   \\ \hline
				100000000 & suma = 1600000000 = 0x5f5e1000 hex  \\ \hline
				200000000 & suma = 3200000000 = 0xbebc2000 hex  \\ \hline
				300000000 & suma = 4800000000 = 0x11e1a3000 hex  \\ \hline
				2000000000 & suma = 32000000000 = 0x773594000 hex   \\ \hline
				3000000000 & suma = -20719476736 = 0xfffffffb2d05e000 hex  \\ \hline
				-100000000 & suma = -1600000000 = 0xffffffffa0a1f000 hex   \\ \hline
				-200000000 & suma = -3200000000 = 0xffffffff4143e000 hex   \\ \hline
				-300000000 & suma = -4800000000 = 0xfffffffee1e5d000 hex  \\ \hline
				-2000000000 & suma = -32000000000 = 0xfffffff88ca6c000 hex  \\ \hline
				-3000000000 & suma = 20719476736 = 0x4d2fa2000 hex  \\ \hline
			\end{tabular}
			\caption{Resultado bateria ejercicio 5.3}
			\label{tabla:sencilla}
		\end{center}
	\end{table}
	
	%-----------------------------------------------------------------------	%							5.4
	%-----------------------------------------------------------------------
	
	\begin{table}[htbp]
		\begin{center}
			\begin{tabular}{|l|l|}
				\hline
				Input list (x8) & Output \\
				\hline \hline
				1,2 &  media = 1  resto = 8  \\ \hline
				-1,-2 & media = -1  resto = -8  \\ \hline
				0x7fffffff,0x7fffffff &	media = 2147483647  resto = 0	\\ \hline
				0x80000000,0x80000000 &	media = -2147483648  resto = 0	\\ \hline
				0xffffffff,0xffffffff &	media = -1  resto = 0	\\ \hline
				2000000000,2000000000 &	media = 2000000000  resto = 0	\\ \hline
				3000000000,3000000000 &	media = -1294967296  resto = 0	\\ \hline
				-2000000000,-2000000000 & media = -2000000000  resto = 0	\\ \hline
				-3000000000,-3000000000 & media = -2000000000  resto = 0	\\ \hline
			\end{tabular}
			\caption{Resultado bateria ejercicio 5.4}
			\label{tabla:sencilla}
		\end{center}
	\end{table}
	
	%-----------------------------------------------------------------------
	%							BIBLIOGRAFIA
	%-----------------------------------------------------------------------
	% Referencia a bibliografia			En \cite{Baz}
	% Referencia a figura				La figura (\ref{fig:1})
	% Espacio entre lineas				\vspace{0.06in}
	% Figura con comentario al pie
	%\begin{figure}[htb]
	%	\centering
	%	\includegraphics[width=0.4\textwidth]{./imagenes/1}
	%	\caption{Universidad de Granada.} \label{fig:1}
	%\end{figure}
	%\begin{thebibliography}{99}
	%	\bibitem{Baz} 
	%	\textsc{Bazaraa, M.S., J.J. Jarvis}
	%	\textit{Programacuib}.
	%	\newline
	%	\url{https://www.google.es}	
	%\end{thebibliography}

	


\end{document}